% Indicate the main file. Must go at the beginning of the file.
% !TEX root = ../main.tex

\chapter{Synthesis and Conclusion}
\label{chap:synthesis} 

\section*{How to Write This Chapter}

The Conclusion or Synthesis chapter serves as the culmination of your thesis, bringing together all papers to demonstrate how your cumulative contribution exceeds the sum of individual parts. This chapter outlines future directions and shows how your work advances the field.

Begin with an opening paragraph of 3 to 4 sentences that previews the chapter structure. This paragraph should synthesize your research findings while explicitly linking them to the original research objectives you raised in your introduction. While writing the synthesis, go back and forth between this chapter and your Introduction chapter to ensure consistency in terminology, research questions, and framing.
Situate your work within the broader context of your field and relevant subfields. Focus on several central themes that unify your contributions. Articulate how these threads weave together to form your thesis's cumulative contribution, and delineate clear, actionable pathways for future research and practical applications.

Ask your supervisor(s) for good examples of synthesis chapters from previous theses in your field.

Remember to reference the research questions from your Introduction chapter using the \texttt{\textbackslash ref\{\}} command.

\section{Main Results}
\label{sec:synthesis-main-results}

\subsection*{Writing Your Main Results Section}

This section directly addresses your research questions and forms the core of your synthesis. Organize your subsections either by research question or by thematic contribution, depending on which structure best reveals the relationships between your papers. Consider including a visual summary such as an overview figure to help readers grasp how your contributions fit together.

Begin with a paragraph that describes how your papers collectively chart a progression, whether methodological, theoretical, or empirical. Explain the journey from your starting point to your end point, and specify the application domain you are targeting. Reference a summary figure that illustrates how the individual papers interrelate and contribute to your overarching research objectives and questions.

Here is an example opening paragraph:

The [number] papers collectively chart a [methodological/theoretical/empirical] progression, moving from [starting point] to [end point] tailored for [application domain].
Figure~\ref{fig:overview_results} summarizes the results, illustrating how the individual papers interrelate and contribute to the overarching research objective and questions.

\begin{figure}
    \centering
    \includegraphics[width=0.9\textwidth]{Figures/summary_questions_results.png}
    \caption{Overview of thesis contributions showing how Papers I through III address the research questions and build upon each other. This figure should clearly illustrate the progression and connections between your papers, making explicit how each paper contributes to answering specific research questions. Adapted from \cite{Delucchi2025}.}
    \label{fig:overview_results}
\end{figure}

Create an overview figure that summarizes how your papers relate to each other. Ensure this figure clearly shows the progression and connections between papers, making it easy for readers to understand your research trajectory at a glance.

\subsection*{Addressing Each Research Question}

Now address each research question with its own subsection. Use unnumbered subsections with the \texttt{\textbackslash subsubsection*\{\}} command when your research question titles are long. Reference the question labels from your Introduction chapter to maintain consistency throughout your thesis.

\subsubsection*{[Research Question 1: Full text of your first research question]}

When addressing each research question, begin by briefly describing what the relevant paper did and found. Explain the study's distinguishing features, particularly any key methodological innovations that ensure important outcomes. Describe your validation approach and how it enhanced the robustness of your results, findings, or models. Discuss how you systematically compared your outputs to baseline or alternative approaches. Demonstrate the feasibility, superiority, or advantages of your method, explicitly stating how it overcomes the limitations of conventional approaches.

Here is an example structure:

Paper~I [briefly describe what Paper I did and found]. 
The study was distinguished by its [key methodological feature] to [ensure some property], thereby ensuring that [important outcome]. 
[Describe validation approach] which enhanced the robustness of the [results/findings/models]. 
The resulting [outputs] were systematically compared to [baseline/alternative approaches]. 
This approach demonstrated the [feasibility/superiority/advantages] of [your method] in [capability], overcoming the limitations of [conventional approaches] that typically [describe limitation].

Summarize how this paper addresses the research question, state your key findings and their significance, and compare your approach with existing methods in the field.

\subsubsection*{[Research Question 2: Full text of your second research question]}

For your second research question, explain how the subsequent paper builds upon the foundations established previously. Address the challenges, limitations, or extensions inherent in broader application or scaling. Describe how you formalized key contributions and what capabilities these enabled. Explain the methodological advances you made and any related developments these necessitated, such as software, theory, or experiments. Clarify what you developed and why these developments were necessary. Show how your contributions provided the backbone necessary for application to real-world, complex, or challenging datasets, problems, or scenarios.

Here is an example structure:

Building upon the foundations from Paper~I, Paper~II addressed the [challenges/limitations/extensions] inherent in [broader application or scaling]. 
The authors formalized [key contribution], enabling [capability].
These methodological advances necessitated [related development, e.g., software, theory, experiments]. 
[Describe what was developed and why]. 
By [enabling/supporting/providing] [key features], [your contribution] provided the [type of] backbone necessary for the application to [real-world/complex/challenging] [datasets/problems/scenarios].

Make explicit how this paper builds on previous work, describe the key methodological or theoretical advances, and explain why these advances were necessary for addressing your research question.

\subsubsection*{[Research Question 3: Full text of your third research question]}

For your third research question, describe how this paper leveraged the methodological advancements from previous work. Explain how you applied your methods or framework to specific datasets or problems. Show how the study explicitly modeled, addressed, or investigated key aspects, thereby addressing critical issues that often plague, limit, or affect certain types of studies or applications. Describe the approaches you employed to achieve your goals, and detail any additional methods you conducted. If applicable, describe your data quality measures. Demonstrate the practical utility of your approach by showing how your models or framework not only improved specific metrics but also provided additional benefits, particularly in addressing problematic scenarios.

Here is an example structure:

Paper~III leveraged the methodological advancements from Paper~II by applying [methods/framework] to [describe dataset or problem].
This study explicitly [modeled/addressed/investigated] [key aspects], thus addressing the critical issue of [important challenge] that often [plagues/limits/affects] [type of studies/applications]. 
[Describe approach] were employed to [achieve goal], while [additional methods] were conducted using [techniques]. 
[Describe data quality measures if applicable].
The practical utility of [your approach] was clearly demonstrated as [these models/this framework] not only improved [metric A] by [achieving property], but also [additional benefit] that can arise when [problematic scenario].

Describe how this paper brings everything together, explain your practical validation or comprehensive application, and state your key results and their implications. If you have more research questions, add additional subsections following the same structure.

\section{Overall Contributions}
\label{sec:synthesis-contributions}

\subsection*{Writing Your Overall Contributions Section}

This section steps back from individual papers to see the big picture. Focus on what is new that emerges from combining all papers rather than simply listing what each paper achieved. This section is often organized thematically rather than by paper, allowing you to highlight cross-cutting innovations and synergies.

Begin by characterizing your results according to interrelated themes. Identify the unifying aspects that tie your contributions together. Show how your work collectively demonstrates advantages over prior approaches. A comparison table can be very effective for showing how your contributions advance the field.

Consider to create a comparison table that highlights your key advances. Be specific about what is new in your work versus what existed in prior approaches. This helps readers quickly grasp your contributions.

\subsection{[Thematic Contribution Area 1]}

Name your thematic contribution areas with descriptive titles such as "Unified Methodological Innovations" or "Theoretical Advances" that capture the essence of your contributions.

\subsubsection*{[Specific Contribution A]}

For each specific contribution, use descriptive titles such as "Advancing Interpretable Machine Learning" or "Novel Framework for X" that clearly indicate what you achieved.

Structure each contribution subsection as follows: first describe the problem that existed, then present your solution, and finally demonstrate the impact. Begin by explaining what traditional approaches have achieved in your domain context, citing relevant literature. However, also note the limitations of these approaches, such as limited capabilities or lack of desired properties. Explain how these limitations have hindered, slowed, or prevented desired outcomes.

Then contrast this with your approach. Explain how your framework or method explicitly provides key capabilities. Show how your method distinguishes itself from, enables advantages over, or provides benefits compared to alternative approaches. Describe additional capabilities that strengthen your approach and ensure desired properties or outcomes.
Clearly state the problem, explain your solution, and demonstrate impact with evidence from your papers.

\subsubsection*{[Specific Contribution B]}

For subsequent specific contributions within a thematic area, use descriptive titles such as "Software Engineering for Reproducible Research" or "Scalable Implementation."

Describe how your developments exemplify best practices or innovations in your domain, addressing specific crises, challenges, or problems in the field. Explain the features your system, package, or framework supports and the benefits these features ensure. Highlight notable innovations, such as specific features within your framework or system. Explain how these features allow certain capabilities through particular mechanisms. Demonstrate how your approach addresses challenges, providing evidence or metrics from your papers.

\subsection{[Thematic Contribution Area 2]}

Continue with additional thematic contribution areas. Most theses have 2 to 4 major thematic areas. Use descriptive titles such as "Practical Applications" or "Data Contributions" that clearly convey the nature of your contributions.

\subsubsection*{[Specific Contribution C]}

For each additional contribution area, provide detailed descriptions of your important contributions. Organize your contributions thematically rather than simply by paper. This organization allows you to show how your contributions work together synergistically, creating value that exceeds the sum of individual parts.

\section{Limitations}
\label{sec:limitations}

\subsection*{Writing Your Limitations Section}

Every thesis has limitations, and addressing them honestly demonstrates scientific maturity. Organize this section either by paper or by type of limitation, depending on which structure better serves your narrative. Common limitation categories include data limitations such as size, quality, or scope; methodological constraints such as assumptions, scalability, or generalizability; validation limitations such as limited external validation or specific domains; scope limitations that identify what was not addressed or boundary conditions; and resource constraints such as time, computational resources, or access limitations.

Begin with a brief acknowledgment that your thesis, while making significant contributions, has several limitations that warrant discussion.

\subsection{[Limitation Category 1]}

For each limitation category, use descriptive titles such as "Data Limitations" or "Methodological Constraints." Describe the limitation clearly and explain why it exists. Was it a deliberate choice, a practical constraint, or a theoretical boundary? Discuss the potential impact of this limitation on your results or conclusions. If applicable, explain how future work could address this limitation, thereby connecting your limitations to future research directions.

\subsection{[Limitation Category 2]}

Continue with additional limitation categories using descriptive titles such as "Computational Constraints" or "Scope Limitations." Follow the same structure as for the first category: describe the limitation, explain why it exists, discuss its potential impact, and if applicable, suggest how future work could address it.

Aim to describe 3 to 5 key limitations across your work. For each limitation, explain its impact on your findings and discuss potential mitigation strategies. This balanced treatment shows that you understand the boundaries of your work while still maintaining confidence in your contributions.

\section{Future Research Directions}
\label{sec:future-work}

\subsection*{Writing Your Future Research Directions Section}

Future work should be concrete and actionable, providing clear pathways for others to build on your research. Distinguish between three types of future work: extensions that represent natural next steps, improvements that address limitations identified in your work, and new directions inspired by your findings. Balance ambitious vision with practical next steps to make your future work both inspiring and achievable.

Begin with a paragraph that introduces the promising avenues your findings have opened, spanning areas such as methodological refinements, broader applications, and fundamental extensions.

\subsection{[Future Direction 1]}

For each future research direction, use descriptive titles such as "Methodological Extensions" or "Integration with Complementary Approaches." Describe the future direction clearly, explaining how it builds on the foundation, framework, or methods established in your thesis. Specify what capabilities this extension would enable and what current limitations or new opportunities it would address. Provide concrete next steps that make the future work actionable.

\subsection{[Future Direction 2]}

Continue with additional future research directions using descriptive titles such as "Applications to New Domains" or "Clinical Translation." Follow the same structure for each direction, making the work concrete and actionable.

\section{Broader Impact}
\label{sec:broader-impact}

\subsection*{Writing Your Broader Impact Section}

This section is optional but increasingly expected in modern theses. It addresses important questions such as: Who benefits from your research? What are the societal implications? What are the ethical considerations? A thoughtful broader impact section demonstrates that you understand your work's implications beyond the immediate scientific community.

Begin by stating that beyond the immediate scientific contributions, your work has broader implications for specific domains, society, or your field.

\subsection{[Impact Area 1]}

For each impact area, use descriptive titles such as "Clinical Decision-Making" or "Policy Implications." Describe the real-world impact or applications of your research. Discuss who benefits from your work and how they benefit. Be specific about the mechanisms through which your research creates value or addresses important problems.

\subsection{[Impact Area 2]}

For additional impact areas, consider topics such as "Ethical Considerations" or "Open Science." Discuss ethical implications, potential biases, or responsible use of your methods and findings. Address how your work promotes responsible research practices. Be honest about both positive impacts and potential negative consequences or misuse.

Discuss positive impacts on practice, policy, or society. Address potential negative consequences or misuse of your research. Describe how your work promotes responsible research practices, such as through open science, reproducibility, or ethical guidelines.

\section{Concluding Remarks}

\subsection*{Writing Your Concluding Remarks}

Your concluding remarks should consist of 1 to 2 paragraphs that tie everything together. End on a forward-looking, inspiring note that leaves readers with a clear sense of your work's significance and future potential.

In the first paragraph, provide a high-level summary of your main achievement. Explain how your key contributions address fundamental challenges in your field related to specific problems. Describe how the integration of your methodological approach with your application domain has yielded key innovations that advance the field.

In an optional final sentence, connect your work to the bigger picture. Consider how your research embodies the spirit of scientific progress, demonstrating that the value of your domain lies not only in specific metrics but in deeper values such as clarity, transparency, trust, or impact that it brings to stakeholders, your field, or society.

\section*{Alternative Organizational Structures}

You have flexibility in organizing your Conclusion chapter. Here are three common structures:

\textbf{Structure A (Question-based):} This structure organizes your synthesis around your research questions. It includes Main Results organized by research questions, Overall Contributions, Limitations, Future Work, and Broader Impact.

\textbf{Structure B (Theme-based):} This structure organizes your synthesis around thematic contributions. It includes Key Findings organized by themes, Methodological Contributions, Practical Contributions, Limitations and Future Work, and Impact and Outlook.

\textbf{Structure C (Paper-synthesis):} This structure emphasizes how your papers build on each other. It includes Paper-by-Paper Summary, Cross-Cutting Themes, Cumulative Contribution, Limitations, and Future Directions.

Choose your structure based on how your papers relate to each other, whether questions or themes emerge more naturally from your work, and the conventions in your university or field.
